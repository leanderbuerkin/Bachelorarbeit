\chapter*{Abstract}
2020 and 2021 have been marked by COVID-19 and the ensuing uncertainty:
How does the virus spread? Does the virus spread from the cities to rural areas? Can the counties whose case numbers skyrocketed earlier be categorized in any way? Or more generally: is the spread of Corona correlated with certain characteristics of a given area?

An attempt to answer these questions and to summarize the given Corona data is described in this thesis. In particular, the project uses data provided by the Robert Koch Institute (RKI) via ArcGIS.

On the basis of the processed data, an SIR model for Germany is created and several correlation analyses of Corona data from various administrative districts are carried out.
Finally, the results are plotted and interpreted in order to make the elusive coronavirus infection patterns accessible.

\chapter*{Kurzfassung}
Die Jahre 2020 und 2021 waren vom Coronavirus und der einhergehenden Unsicherheit geprägt:
Wie verbreitet sich das Virus? Breitet sich das Virus von den Städten ausgehend in die ländlichen Gebiete aus? Lassen sich die Stadt- und Landkreise, deren Fallzahlen früher in die Höhe schießen, irgendwie kategorisieren? Oder ganz allgemein: Korreliert die Ausbreitung von Corona mit bestimmten Eigenschaften der Gebiete?

Ein Versuch, diese Fragen zu beantworten und die gegebenen Corona-Daten zusammenzufassen, ist in dieser Arbeit beschrieben. Insbesondere wird mit Daten, welche das Robert Koch-Institut (RKI) über ArcGIS zur Verfügung stellt, gearbeitet.

Auf Basis der aufgearbeiteten Daten wird ein SIR-Modell für Deutschland erstellt und mehrere Korrelationsanalysen von Corona-Daten die kreisfreien Städte, Landkreise und Regierungsbezirke durchgeführt.
Zuletzt werden graphische Darstellungen der Ergebnisse erstellt und interpretiert, um das schwer greifbare Infektionsgeschehen des Coronavirus zugänglich zu machen.
\newpage