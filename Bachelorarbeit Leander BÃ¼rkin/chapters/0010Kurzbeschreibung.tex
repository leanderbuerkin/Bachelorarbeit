\chapter*{Abstract}
\chapter*{Kurzfassung}
Die Jahre 2020 und 2021 waren vom Coronavirus und der einhergehenden Unsicherheit geprägt - Wie verbreitet sich das Virus? Breitet sich das Virus von den Städten aus aus oder lassen sich die Landkreise, deren Fallzahlen früher in die Höhe schießen, irgendwie kategorisieren? Ganz allgemein: Ist die Ausbreitung irgendwie rekonstruierbar?

Ein Versuch, diese Fragen zu beantworten und die gegebenen Corona-Daten zusammenzufassen, ist in dieser Arbeit beschrieben. Im speziellen werden die Daten, welche das Robert-Koch-Institut über ArcGIS zur Verfügung stellt, mit mathematischen Formeln analysiert, die Ergebnisse graphisch dargestellt und interpretiert.

Durch die Aufarbeitung dieser Daten wird aus allen vorliegenden COVID-19 Daten der deutschen Landkreise ein leicht zugängliches Bild erzeugt, welches schwer greifbare Prozesse visualisiert und mit einfachsten Mitteln der Mathematik einige mögliche Abhängigkeiten nachprüft.

Hierzu werden die vom Robert Koch-Institut (RKI) öffentlich bereitgestellten Daten zur COVID-19 Pandemie verwendet: Zu jedem deutschen Landkreis liefert das RKI die akkumulierten Infektionszahlen für jeden Tag seit dem 01.03.2020, die Fläche, die Position, den Gemeindeschlüssel sowie die von den statistischen Landesämtern geschätzte Einwohnerzahlen am 31.12.2019. Diese Einwohnerzahlen werden in Deutschland zur Berechnung der 7-Tage-Inzidenz (Neuansteckungen pro 100.000 Einwohnern in 7 Tagen) verwendet. \todo{Quelle?}
Zudem werden alle Meldungen der Gesundheitsämter genutzt, welche Aufschluss darüber geben, wieviele Menschen wo und wann infiziert wurden und nach welchem Zeitraum sie leider gestorben oder glücklicherweise genesen sind.
\todo{Quelle angeben}


Aus diesen Daten lassen sich folgende weitere Größen für die einzelnen Landkreise berechnen:
\begin{itemize}
    \item Bevölkerungsdichte
    \item 7-Tages-Inzidenzen
    \item Korrelationswahrscheinlichkeiten zwischen zwei 7-Tages-Inzidenzen mit einer zeitlichen Verschiebung um bis zu 50 Tage
    \item Das Verhältniss von noch nie Infizierten (S), Infizierten (I) und aus dem Infektionsgeschehen entfernten Personen (R bzw. R und D, wenn Tote gesondert aufgeführt werden) - dies entspricht dem SIR-Modell
    \todo{checken, ob alle hier aufgeführten Abbildung in der finalen Arbeit stecken.}
\end{itemize}

Diese Daten werden mithilfe der Gemeindeschlüssel auch für die deutschen Bundesländer und Regierungsbezirke berechnet.

\todo{Ergebnisse in Zusammenfassung hinzufügen}
\todo{Abstract übersetzen}
\newpage