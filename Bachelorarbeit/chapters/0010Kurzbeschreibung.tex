\chapter*{Abstract}
\chapter*{Kurzfassung}
Die Jahre 2020 und 2021 waren vom Coronavirus und der einhergehenden Unsicherheit geprägt - Wie verbreitet sich das Virus? Wie schnell entstehen Hotspots? Ganz allgemein: Ist die Ausbreitung irgendwie einfach quantifizierbar?

Ein Versuch, diese Fragen zu beantworten und die gegebenen Corona-Daten zusammenzufassen, ist in dieser Arbeit beschrieben. Im speziellen werden die Daten, welche das Robert Koch-Institut über ArcGIS zur Verfügung stellt, mit mathematischen Formeln der Diffusion analysiert.


Zu jedem deutschen Landkreis liefert das Robert Koch-Institut die akkumulierten Infektionszahlen seit dem 01.03.2020, die Fläche, die Position sowie die von den statistischen Landesämtern geschätzte Einwohnerzahle am 31.12.2019, welche in Deutschland zur Berechnung der 7-Tage-Inzidenz (Neuansteckungen pro 100.000 Einwohnern in 7 Tagen) benutzt wird.


Aus diesen Daten lässt sich der Verlauf folgender Größen darstellen:
\begin{itemize}
    \item Akkumulierte Fallzahlen eines Landkreises
    \item Das Verhälniss von Ansteckbaren (S), Infizierten (I) und aus dem Infektionsgeschehen entfernten Personen (R bzw. R und D, wenn Tote gesondert aufgeführt werden) - dies entspricht dem SIR-Modell
    \item 7-Tage-Inzidenz eines Landkreises
    \item 7-Tage-Inzidenzen aller Landkreise
    \item 7-Tage-Inzidenzen aller Landkreise im Vergleich zur Bevölkerungsdichte des Landkreises
    \todo{checken, ob alle hier aufgeführten Abbildung in der finalen Arbeit stecken.}
\end{itemize}

\todo{Das folgende wurde Aus Motivation hierher verschoben, bitte einpflegen}

Ziel dieser Arbeit ist es, aus allen vorliegenden COVID-19 Daten der deutschen Landkreise ein leicht zugängliches Bild zu erzeugen, welches schwer vorstellbare Prozesse visualisiert und mit einfachsten Mitteln der Mathematik einige mögliche Abhängigkeiten nachzuprüfen.

\todo{Ergebnisse in Zusammenfassung hinzufügen}
\todo{Abstract übersetzen}
\newpage