\chapter{Motivation}\label{chap:Motivation}
Viele große Zivilisationen hatten bereits mit Pandemien ähnlich zur aktuellen COVID-19 Pandemie zu kämpfen.\\
Schon aus dem Alten Rom haben wir Zeugnisse der Antoninischen Pest im zweiten Jahrhundert nach Christus: Vermutlich ein Pocken Ausbruch, dessen 7-10 Millionen Tote das Römische Reich destabilisierten (\autocite{RomPest}, S. 255).
Im Mittelalter kamen 200 Millionen Menschen durch die Pest um \autocite{PestMittelalter}.
Die Armeen des ersten Weltkriegs wurden ebenfalls durch die Spanische Grippe geschwächt: Sie forderte circa 100 Millionen Opfer weltweit \autocite{SpanischeGrippe}.

Doch nicht nur in den dicht besiedelten Gebieten des Mittelmeerraums und des Nahen Ostens sorgten Pandemien für Instabilität: Sie sind seit jeher ein weltweites Problem und können durch Reisen ausgleöst oder stark verstärkt werden.\\
Nach 1707 starben innerhalb von zwei Jahren circa 18.000 der 50.000 Einwohner Islands an Pocken (\autocite{americaPandemics}, Seite 325 Zeile 20ff). Vieles spricht für einen ähnlich vernichtenden Pockenausbruch bei den Azteken, den Inka und den andere Völkern der Neuen Welt \autocite{americaPandemics}.

Über diese vergangenen Pandemien wissen wir recht wenig, da sie beispielsweise durch den Zorn Gottes erklärt wurden(\autocite{americaPandemics}, Seite 324 Zeile 7) oder die genannten Zahlen eher die Gefühle der Zeitzeugen widerspiegeln als statistische Daten zu liefern (\autocite{americaPandemics}, Seite 324 Zeile 20ffff). Mit dem heutigen Stand der Datenerfassung und -übermittlung lässt sich die COVID-19 Pandemie sehr viel einfacher untersuchen als ihre Vorgänger.

um derartige Todeswellen zu vermeiden und einer Zivilisation Stabilität zu bieten, ist es wichtig, diese einschneidenden Ereignisse mit verschiedensten Ansätzen zu untersuchen. 

Die meisten von uns haben genau diese Zahlen des Robert-Koch-Instituts oder der Weltgesundheitsorganisation beobachtet und versucht zu verstehen, wie man sich morgen zu Verhalten hat. Viele Hypothesen wurden aufgestellt und jede*r musste sich auf einmal mit Epidemiologie beschäftigen, wobei wir meist nur ein Puzzleteil vor Augen hatten.

Trotz immenser Bemühungen brachte die aktuelle COVID-19 Pandemie unser Gesundheitssystem an seine Grenzen: Kleinere Krankenhäuser wurden schnell überwältigt, wenn der Bedarf an Beatmungsgeräten größer wurde als ihre Kapazitäten.

Daher soll diese Arbeit das gesellschaftliche Verständniss für zukünftigen Pandemien schärfen, damit die Ausbreitung zukünfitger Erreger verlangsamt wird und ein Kollaps des Gesundheitssystems oder gar der gesamten Zivilisation verhindert werden kann.
