\subsection{Korrelationsanalyse aller Landkreise}
Wie in \autoref{sec:Korrelationsanalyse} beschrieben wird die Korrelation zwischen jedem möglichen Paar von Landkreisen gebildet. \todo{Datei verlinken}

Die Verschiebungen mit der höchsten Wahrscheinlichkeit einer Korrelation werden in Abbildung \autoref{} dargestellt. Da bei einer größeren Verschiebung immer weniger Produkte aufsummiert werden und schlussendlich durch die Anzahl der Produkte geteilt wird, fallen einzelne Ausbrüche stärker ins Gewicht, je größer die Verschiebung ist. Daher werden nur die Matrizen für eine maximale Verschiebung $\tau$ von $-50$~Tage $\leq\tau\leq50$~Tage, bzw. $-30$~Tage $\leq\tau\leq30$~Tage, $-14$~Tage $\leq\tau\leq14$~Tage.

Zudem wird eine vierte Matrix dargestellt, für die die Wahrscheinlichkeiten für eine Korrelation für jede Verschiebung addiert werden. Die Wahrscheinlichkeiten für negative Verschiebungen werden mit negativem Vorzeichen versehen, die  Wahrscheinlichkeiten für positive Verschiebungen werden mit positivem Vorzeichen versehen. Die Wahrscheinlichkeit für eine Verschiebung $\tau=0$ wird nicht berücksichtigt. 
Da alle Wahrscheinlichkeiten positiv sind, da die jeweils multiplizierten Zahlen der Corona-Fälle positiv sind, ergibt sich so eine Gesamtwahrscheinlichkeit, inwiefern die Corona Zahlen des einen Landkreises den Corona Zahlen des anderen Landkreises folgen. Es lässt sich jedoch nicht mehr herausfinden, um wieviele Tage die Kurve wahrscheinlich verschoben ist, sondern nur noch ob eher negativ oder positiv.

Zählt man schlussendlich die Werte einer Zeile der Matrix zusammen lässt sich bestimmen, ob die Corona Fälle des Landkreises im nationalen Vergleich eher früher oder später steigen und welcher Fall wie wahrscheinlich ist. Diese Werte werden auf einer Deutschland Karte farblich dargestellt.

