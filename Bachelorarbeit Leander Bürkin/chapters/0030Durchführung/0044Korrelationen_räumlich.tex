\section{Durchschnittliche Verschiebung im nationalen Vergleich}
\subsection{Verteilung der durchschnittlichen Verschiebung unter den Landkreise}
In den folgenden Abbildungen sind die deutschen Landkreise entsprechend dem Durchschnitt der Werte ihrer Zeile in der beschriebenen Matrix eingefärbt.


In \autoref{fig:average_shift_counties.png} sind die Landkreise entsprechend der Durchschnitte ihrer Zeilen in der rechten unteren Matrix aus \autoref{fig:matrizes_pop_density_counties} bzw. \autoref{fig:matrizes_north_to_south_counties} eingefärbt. Links daneben ist die Verteilung dargestellt.

\begin{figure}[H]
    \centering
    \includegraphics[width = 0.95\textwidth]{figures/Ergebnisse/average_shift_counties.png}
    \caption{Auf der rechten Seite die deutschen Landkreise eingefärbt im Durchschnitt der Maximalwerte der Korrelationen mit allen Landkreisen und Verschiebungen $\tau\in[-50,50]$. Auf der linken Seite die Verteilung dieser Werte.}
    \label{fig:average_shift_counties.png}
\end{figure}

Die Landkreise in \autoref{fig:positive_or_negative_shift_counties} sind  entsprechend der Durchschnitte ihrer Zeilen in der linken unteren Matrix aus \autoref{fig:matrizes_pop_density_counties} bzw. \autoref{fig:matrizes_north_to_south_counties} eingefärbt. Die Verteilung ist links daneben dargestellt.

\begin{figure}[H]
    \centering
    \includegraphics[width = 0.95\textwidth]{figures/Ergebnisse/positive_or_negative_shift_counties.png}
    \caption{Auf der rechten Seite die deutschen Landkreise eingefärbt im Durchschnitt der Tendenzen der Verschiebung der Korrelationen dieses Landkreises mit allen Landkreisen und Verschiebungen $\tau\in[-50,50]$. Auf der linken Seite die Verteilung dieser Werte.}
    \label{fig:positive_or_negative_shift_counties}
\end{figure}



\subsection{Verteilung der durchschnittlichen Verschiebung unter den Regierungsbezirken}
Die deutschen Regierungsbezirke sind in den folgenden Abbildungen entsprechend dem Durchschnitt der Werte ihrer Zeile in der referenzierten Matrix eingefärbt.

In \autoref{fig:average_shift_districts.png} sind die Regierungsbezirke entsprechend der Durchschnitte ihrer Zeilen in der rechten unteren Matrix aus \autoref{fig:matrizes_pop_density_districts} bzw. \autoref{fig:matrizes_north_to_south_districts} eingefärbt. Links daneben ist die Verteilung dargestellt.

\begin{figure}[H]
    \centering
    \includegraphics[width = 0.95\textwidth]{figures/Ergebnisse/average_shift_districts.png}
    \caption{Auf der rechten Seite die deutschen Regierungsbezirke eingefärbt im Durchschnitt der Maximalwerte der Korrelationen mit allen Regierungsbezirken und Verschiebungen $\tau\in[-50,50]$. Auf der linken Seite die Verteilung dieser Werte.}
    \label{fig:average_shift_districts.png}
\end{figure}

Die Regierungsbezirke in \autoref{fig:positive_or_negative_shift_districts} sind  entsprechend der Durchschnitte ihrer Zeilen in der linken unteren Matrix aus \autoref{fig:matrizes_pop_density_districts} bzw. \autoref{fig:matrizes_north_to_south_districts} eingefärbt. Die Verteilung ist Links daneben dargestellt.

\begin{figure}[H]
    \centering
    \includegraphics[width = 0.95\textwidth]{figures/Ergebnisse/positive_or_negative_shift_districts.png}
    \caption{Auf der rechten Seite die deutschen Regierungsbezirke eingefärbt im Durchschnitt der Differenz der Korrelationswerte der positiven Verschiebungen zu den Korrelationswerten der negativen Verschiebungen der Korrelationen dieses Regierungsbezirks mit allen Regierungsbezirken und Verschiebungen $\tau\in[-50,50]$. Auf der linken Seite die Verteilung dieser Werte.}
    \label{fig:positive_or_negative_shift_districts}
\end{figure}

