\chapter{Motivation}\label{chap:Motivation}
Viele große Zivilisationen vor uns hatten bereits mit Pandemien ähnlich zur aktuellen COVID-19 Pandemie zu kämpfen.\\
Schon aus dem Alten Rom haben wir Zeugnisse der Antoninischen Pest im zweiten Jahrhundert nach Christus: Vermutlich ein Pocken Ausbruch, dessen 7-10 Millionen Tote das Römische Reich destabilisierten (\autocite{RomPest}, S. 255).
Im Mittelalter kamen 200 Millionen Menschen durch die Pest um \autocite{PestMittelalter}.
Auch die Armeen des ersten Weltkriegs wurden durch eine Pandemie stark geschwächt: die Spanische Grippe forderte circa 100 Millionen Opfer weltweit \autocite{SpanischeGrippe}.

Daten in Relation zur Bevölkerung haben wir beispielsweise aus Island: nach 1707 starben dort innerhalb von zwei Jahren ein Drittel, also circa  18.000 der 50.000 Einwohner, an Pocken (\autocite{americaPandemics}, S.325%, Zeile 20ff
). Die Auswirkungen auf die Azteken, die Inka und die anderen Völker der Neuen Welt, welche vermutlich ein ähnlich vernichtender Pockenausbruch traf, und ihr anschließender Niedergang sind uns ebenfalls wohl bekannt \autocite{americaPandemics}.

Um derartige Schicksale in unserer globalisierten Welt zu verhindern ist es unabdingbar, diese Bedrohungen bestmöglich zu untersuchen. Denn, wie eingangs erwähnt, \glqq{}wer seine Geschichte nicht kennt, ist dazu verdammt, sie zu wiederholen\grqq{} (Freie Übersetzung des Zitats von George Santayana \autocite{history-quoteSantayana}).
\autoref{fig:spanishflu} und \autoref{fig:germany_incidence} zeigen, dass wir noch viel zu lernen haben, damit sich der Verlauf einer Pandemie nicht wiederholen.


Jedoch wissen wir über diese vergangenen Pandemien recht wenig, da nicht nach heutigen wissenschaftlichen Standards Daten erhoben und Schlussfolgerungen gezogen wurden. Beispielsweise wurden die Pandemien durch den Zorn Gottes erklärt (\autocite{americaPandemics}, S. 324% Zeile 7
) und die Zahlen der Zeitzeugen spiegeln eher die subjektive Wahrnehmung der Zeitzeugen wieder als wissenschaftliche Daten zu liefern (\autocite{americaPandemics}, S. 324% Zeile 20-24
). Mit dem heutigen Stand der Datenerfassung und -übermittlung lässt sich die COVID-19 Pandemie sehr viel einfacher untersuchen als ihre Vorgänger.

Die meisten von uns haben genau diese Zahlen des Robert Koch-Instituts (RKI, \href{www.rki.de}{www.rki.de}) oder der Weltgesundheitsorganisation (\href{www.who.int}{www.who.int}) beobachtet und versucht zu verstehen, wie man sich zu Verhalten hat. Viele Hypothesen wurden aufgestellt und man musste sich als Laie auf einmal mit Fragen und Begriffen der Epidemiologie beschäftigen, wobei man meist nur ein Puzzleteil vor Augen hatte.

Daher soll diese Arbeit zum Verständnis von Pandemien beitragen, damit die Ausbreitung zukünftiger Erreger verlangsamt wird und der Kollaps des Gesundheitssystems oder der gesamten Zivilisation gar nicht erst im Bereich des Möglichen liegen.